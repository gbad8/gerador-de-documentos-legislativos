\documentclass[12pt]{letter} %documento tipo carta letra tam. 12
\usepackage{ragged2e}
\usepackage{fancyhdr}
\usepackage[utf8]{inputenc}
\usepackage{antiqua}
\usepackage[T1]{fontenc}
\usepackage{textcomp}
\usepackage{tgtermes}
\usepackage[portuguese, provide=*]{babel}
\usepackage{graphicx} %pacote para insercao de imagens
\usepackage[hidelinks]{hyperref} %pacote para insercao de links
\usepackage{mfirstuc} %capitaliza primeira letra da frase
\setlength{\parindent}{1,5cm} %recuo de paragrado
\setlength{\parskip}{0pt} %distancia entre paragrafos
\usepackage{geometry}

\geometry{
    a4paper, %tipo de papel a4
    top=3cm, %margem de cima
    bottom=5cm %margem de baixo
}

%configuracoes gerais cabecalho e rodape
\pagestyle{fancy}
\setlength{\headheight}{112.2pt} %altura do cabeçalho
\setlength{\headsep}{0.5cm} %distância entre o cabeçalho e o corpo
\fancyhf{} %limpa os campos de cabecalho e rodape
\renewcommand{\headrulewidth}{0.4pt} %linha divisoria cabecalho
\renewcommand{\footrulewidth}{0.4pt} %linha divisoria rodape

%cabecalho centralizado com brasao e texto institucional
\fancyhead[C]{
  \begin{minipage}[c]{0.9\textwidth}
    \centering
    \includegraphics[width=1.8cm]{static/img/brasao.png} \\[0.5em]
    \textbf{\footnotesize ESTADO DO MARANHÃO} \\[-0.2em]
    \textbf{\footnotesize PODER LEGISLATIVO} \\[-0.2em]
    \textbf{\footnotesize CÂMARA MUNICIPAL DE VILA NOVA DOS MARTÍRIOS}
    \vspace{0,7em}
  \end{minipage}
}

%rodape com endereço e contato
\setlength{\footskip}{25pt} %distancia para o conteudo da page
\fancyfoot[C]{
\begin{minipage}{\textwidth}
    \vspace{0,4em}
    \centering
    {\fontsize{8}{11}\selectfont
    \textbf{Av. Rio Branco s/nº Centro, CEP: 65.924-000.} \\
  \href{https://www.cmvilanovadosmartirios.ma.gov.br/}{\textbf{https://www.cmvilanovadosmartirios.ma.gov.br/}}\\
    \textbf{Email:} \href{mailto:cmvnmartirios@hotmail.com}{\textbf{mvnmartirios@hotmail.com}} \\
    }
  \end{minipage}}

% preambulo do documento
\begin{document}
\begin{flushright} %alinhado a direita
\textbf{ Secretaria Legislativa } \\ %orgao inserido pelo user

%identificacao do oficio
\begin{flushleft} %alinhado a esquerda
\textbf{Indicação n° {{numero}}/{{ano}}} %numero do oficio inserido pelo user
\end{flushleft}
\vspace{0.6cm} %espaco entre orgao e data
Vila Nova dos Martírios, {{data}}. %data inserida pelo user
\end{flushright}

% autoria
\vspace{0.6cm}
\begin{flushleft} %alinhando a esquerda
\textbf{AUTORIA \MakeUppercase{ {{vereador}} }}
\end{flushleft}
\vspace{0.3cm}

% assunto da indicacao
\begin{justify}
\textbf{ASSUNTO: \MakeUppercase{ {{assunto}} }}
\end{justify}
\vspace{0.3cm}

%dedicatoria
\begin{justify}
	Ao Exmo. Sr. Vereador Josemar Rodrigues da Silva, \\
	Presidente da Câmara Municipal de vereadores \\
\end{justify}

%preambulo
\begin{justify}
	Em concordância ao Regimento Interno da Câmara Municipal de Vereadores de Vila Nova dos Martírios (Art. 140) e por ser um legítimo Representante do povo desta municipalidade, faço a seguinte indicação para leitura, discussão e votação em Plenário:
\end{justify}

%solicitação
\vspace{0.3cm}
\begin{justify}
	\MakeUppercase{\textbf{SOLICITAÇÃO: {{solicitação}} }}
\end{justify}

% justificativa
\vspace{0.3cm}
\begin{justify}
	\textbf{JUSTIFICATIVA:} \capitalisewords{ {{justificativa}} }
\end{justify}

%local e data
\vspace{0.3cm}
\begin{justify}
		Sala das Sessões da Câmara Municipal de Vereadores de Vila Nova dos Martírios - MA Plenário Aulindo Batista da Cruz. {{data}}
\end{justify}

%assinaturas
\vspace{1.3cm}

\begin{center}
\rule{8cm}{0.4pt} \\[1ex]
\textbf{ {{vereador}} } \\[0.5ex]
\textit{ {{vereador-autor}} }
\end{center}

\end{document}